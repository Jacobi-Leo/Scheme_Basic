\documentclass[]{article}
\usepackage[a4paper]{geometry}
\usepackage{minted}
\usemintedstyle{bw}
\usepackage{amsmath}
\usepackage{bm}
\usepackage{siunitx}
\usepackage{color}
\usepackage{fancyhdr}
\usepackage{float}
\usepackage{subcaption}
%\usepackage{eumeitem}
\usepackage[]{graphicx}
%
%% \num{10e2}
%% \mmHg
%% \SI{4}{\milli\meter}
%% \gg  % instead of $>\!\!>$
%

\pagestyle{plain}
%\graphicspath{{}}
\bibliographystyle{plain}

\newcommand{\me}{\mathrm{e}}
\newcommand{\mi}{\mathrm{i}}
\newcommand{\mRe}{\mathit{Re}}
\newcommand{\mSt}{\mathit{St}}
\newcommand{\mRo}{\mathit{Ro}}
\newcommand{\Karman}{K\'arm\'an}
%\newcommand{\diff}{\,\mathrm{d}}
\newcommand{\diff}{d}

\input{texhead.h}

\title{The Grid Average and Nodal Value in High Order Scheme}
\author{Qingdong Cai, Zeyu Liu}
\date{\today}

\begin{document}
\maketitle
\begin{abstract}
    Strict distinction between nodal value and grid average in constructing
    numerical schemes is made in this work and some common faults in
    misidentifying the two concepts are pointed out. 
    The expansion based on grid average,
    similar to Taylor's expansion on nodal value, is deduced to construct
    correct scheme in terms of grid average and to obtain modified equation
    which determines the order of accuracy of numerical scheme theoretically.
    The correct high order scheme, QUICK scheme, is constructed in different
    approaches, the property of interpolation coefficients is analyzed, and
    the absolute error of high order schemes is investigated numerically. 
\end{abstract}
\tableofcontents
\newpage

\section{Introduction}
\label{sec:intro}

The numerical scheme can be classified in terms of accuracy. To a 
specific scheme, the order of accuracy is defined as the order of leading term
in truncation error of the corresponding modified equation. Heuristic estimates on
order of accuracy is usually given by observing the orders of leading terms in
each individual term's truncation error in the discretization process of the
primary PDE, and the lowest order among all terms is selected as the order of that
scheme. This method provides exact result only for first- and second-order 
schemes and fails in high order schemes such as Quadratic Upwind Interpolation
Convective Kinematics (QUICK) scheme \cite{leonard1979stable}.
It is a common fault that the result of heuristic analysis is treated as exact
order of accuracy of high order numerical scheme.

There are two types of data with which numerical schemes are constructed: nodal
value and grid average, and the confusion between the two concepts is another
common fault. To a specific function, whether scalar or vectorial,
nodal values are defined as values of the function at grid node points,
and grid averages are defined as normalized integrate of the function on each
grid element.
While constructing first- and second-order schemes, piecewise constant
approximation and piecewise linear approximation are assumed respectively.
In those cases, value at the center of grid is equivalent to integral average 
over the grid element, so it is not necessary to distinguish nodal value and grid
average in constructing of numerical schemes. However, in high order schemes
(higher than second-order), nodal value and grid average are inconsistent,
so the expected order of accuracy is achieved only if strict distinction is made.
The more wildly used data is nodal value since Taylor's
expansion can be applied naturally in Finite Difference Method (FDM), however
grid average is more convenient in Finite Volume Method (FVM) since its
conceptual accordance with conservation laws. 
In Sec.~\ref{sec:expansion-based-grid}, similar to Taylor's expansion, a new
method of expansion of grid average over different orders of derivatives of the
function is established, which is the theoretical basis of the following
analysis. 
When FVM schemes are constructed, grid averages are naturally introduced, but 
one common fault is that nodal value is regarded as grid
average and Taylor's expansion is directly applied to discrete data, 
which produces the same form of scheme in first- and
second-order cases (the detail materials are provided in 
Sec.~\ref{sec:first-order-upwind},~\ref{sec:second-order-central},~and~\ref{sec:lax-wendroff-scheme}),
but fails to produce higher order
schemes in the same way. Taking QUICK scheme as example, we present how to
introduce grid average into
construction of numerical scheme correctly, and the counterpart with nodal value
as basic variable is also discussed.
Furthermore, the relationship between interpolation
coefficients of different orders schemes is attained in
Sec.~\ref{sec:prop-coeff-grid-avg},~and~\ref{sec:prop-coeff-nodal-value}.

In Sec.~\ref{sec:abserr-high-order}, the relationship between
absolute error and grid size is investigated by numerical test, the results
implies that for a specific scheme, there exists one optimum size:
truncation error dominates the absolution error when grid size is larger
than the optimum, and round-off error dominates the absolute error when
grid size is smaller than the optimum. Since trunctation error declines while
grid size shrinks, and round-off error increases while grid size shrinks, the
absolute error tends to mimimum when grid size is set to that optimum.



\section{Expansion Based on Grid Average}
\label{sec:expansion-based-grid}

The Finite Difference Method (FDM) is in terms of nodal values,
so it is convenient to apply Taylor's expansion to FDM. As a result,
almost all widely used numerical schemes are based on Taylor's expansion. 
However, in FVM, the basic variables are grid averages.  FVM schemes should
be constructed with expansions based on grid averages
instead of applying Taylor's expansion expansion directly. 
In this section, a new method is presented to achieve expansion based on grid
averages.

We take 1-D convection equation as example:
\begin{equation}
    \dfrac{\partial u}{\partial t}+a\dfrac{\partial
    u}{\partial x}=0,\hspace{1cm}x\in [0,L],\hspace{5mm} t\in [0,T].
    \label{eq:mod-e1} 
\end{equation}
rewrite Eq.~(\ref{eq:mod-e1}) into conservative form:
\begin{equation}
    \dfrac{\partial u}{\partial t}+\dfrac{\partial f
    }{\partial x}=0,
    \label{eq:conservative-convection}
\end{equation}
while the flux function $f(x) =au$ produces the linear equation in
Eq.~(\ref{eq:mod-e1}), and other PDE is attained with different flux
function such as $f(u) = \frac 12 u^2$ produces a non-linear equation.

Given initial data of Eq.~(\ref{eq:mod-e1}), the solution is
\begin{equation*}
    u = u(x,t) = u_0(x-at).
\end{equation*}
Discretize Eq.~(\ref{eq:mod-e1}) in space-time $(x, t)$ with uniform grid,
then the coordinates of grid nodes are:
\begin{align*}
    x_j=j \Delta x,\hspace{1cm} j=0,1,2,\ldots,N;\hspace{1cm} \Delta x
    =\frac{L}{N}. \\
    t^n = n\Delta t,\hspace{1cm} n=0,1,2,\ldots,M;\hspace{1cm} \Delta t
    =\frac{T}{M}.
\end{align*}
There are two kinds of unknowns: nodal values $u^n_j=u(x_j,t^n)$ in FDM
and grid averages in FVM which is defined as
\begin{equation}
    \bar u_{j}^n = \frac{1}{\Delta x}\int_{x_{\jmh}}^{x_{\jph}}
    u(x,t^n)\diff x.
    \label{eq:def_grid_avg}
\end{equation}
Obviously grid averages differ from nodal values, but the difference
vanishes in first and second-order schemes due to piecewise constant
approximation and piecewise linear approximation. For higher order
schemes, nodal values are different from grid averages, so we have to
distinguish the two concepts clearly to ensure the accuracy of
schemes.

Taylor's expansion can be applied naturally to nodal values, but not for
grid averages $\bar u_j^n$. The antiderivative of $u(x,t)$, 
denoted as $V(x,t)$,
should be introduced to produce the proper expansion. Define the notation
$V_{j+1/2}(t)=V(x_{j+1/2}t)$, then we have
\begin{equation}
    \bar u_{j+k}^{n+\ell}=
    \frac{V_{\jph+k}^{n+\ell}-V_{\jmh+k}^{n+\ell}}{\Delta x}.
    \label{eq:uVrelation}
\end{equation}
Since $V_{\jph+k}^{n+l}$ is nodal value, Taylor's expansion can be
applied to point $(j,n)$:
\begin{equation}
    V_{\jph+k}^{n+\ell} = V_{j}^n + \sum_{m=1}^{\infty}
    \left[ 
	\frac{1}{m!} 
	\left(
	\ell\Delta t \frac{\partial}{\partial t} + 
	\left(k+\frac{1}{2}\right) \Delta x\frac{\partial}{\partial x}
	\right)^m V_j^n
    \right].
    \label{eq:V_taylor}
\end{equation}
Substitute Eq.~(\ref{eq:V_taylor}) into Eq.~(\ref{eq:uVrelation}), we
have:
\begin{equation}
    \bar u_{j+k}^{n+\ell} = \frac{1}{\Delta x} \sum_{m=1}^{\infty} 
    \left[
	\frac{1}{m!}\left( \ell\Delta t\frac{\partial}{\partial t} +
	\left(k+\frac{1}{2}\right)\Delta x\frac{\partial}{\partial x}\right)^m 
	V_j^n -
	\frac{1}{m!}\left(\ell\Delta t\frac{\partial}{\partial t} +
	\left(k-\frac{1}{2}\right)\Delta x\frac{\partial}{\partial x}\right)^m 
	V_j^n
    \right].
    \label{eq:uVrelation2}
\end{equation}
The derivatives to variable $t$ all vanish, and the derivatives of $V$ to
variable $x$ can be replaced by relation $u=V_x$:
\begin{equation}
    \bar u_{j+k}^{n+\ell} = \frac{1}{\Delta x} \sum_{m=1}^{\infty}\frac{1}{m!}
    \left\{
	\sum_{p=1}^m C_m^p(\ell\Delta t)^{m-p} 
	\left[
	    \sum_{s=0}^p \frac{1}{2^s}C_p^s k^{p-s}\left[1-(-1)^s\right]
	\right]
	\Delta x^p 
	\left. 
	\frac{\partial ^{m-1}u}{\partial t^{m-p}\partial x^{p-1}}
	\right|_j^n
    \right\}.
    \label{eq:ubar_taylor}
\end{equation}
Eq.~(\ref{eq:ubar_taylor}) is to some degree Taylor's expansion of
grid average, so substitute it into numerical schemes and we can
attain the modified equation of that scheme.

Consider several special conditions for Eq.~(\ref{eq:ubar_taylor}).
When $\ell = 0, k=0$:
\begin{equation}
    \bar u_j^n = \sum_{m=1}^\infty \frac{1-(-1)^m}{2^m m!} \Delta x^{m-1}
    \left.  \frac{\partial ^{m-1}u}{\partial x^{m-1}} \right|_j^n
    =\sum_{p=0}^{\infty} \frac{1}{4^p(2p+1)!} \Delta x^{2p}
    \left. \frac{\partial ^{2p}u}{\partial x^{2p}}\right|_j^n.
    \label{eq:ubar_taylor_l0k0}
\end{equation}
The second equation in Eq.~(\ref{eq:ubar_taylor_l0k0}) is obtained
because the  even  terms in the summation over  $m$ is definitely zero,
so consider odd terms and substitute $m$ with $m=2p+1$ to transfer the
summation over $m$ to summation over $p$. When $k=0$ is given and
$\ell$ is arbitrary:
\begin{align}
    \bar u_{j}^{n+\ell}
    &\displaystyle
    = \frac{1}{\Delta x}
    \sum_{m=1}^{\infty}\frac{1}{m!}
    \left[
	\sum_{p=0}^m
	(\ell\Delta t)^{m-p}\frac{1-(-1)^p}{2^p}\Delta x^p
	\left. \frac{\partial^m V}{\partial t^{m-p} \partial x^p} \right|_j^n 
    \right]  \label{eq:ubar_taylor_k0p}\\   
    &\displaystyle
    = \sum_{m=1}^{\infty}\frac{1}{m!}
    \left[
	\sum_{s=0}^{\left[\frac{m-1}{2}\right]} \frac{1}{4^s} C_m^{2s+1} 
	(\ell\Delta t)^{m-2s-1}\Delta x^{2s} 
	\left.
	\frac{\partial^{m-1} u}{\partial t^{m-2s-1}\partial x^{2s}}
	\right|_j^n 
    \right], 
    \label{eq:ubar_taylor_k0}
\end{align}
with $\left[\frac{m-1}{2}\right]$ representing floor function of
$\frac{m-1}{2}$. When given $\ell =0$ and $k$ is arbitrary:
\begin{align}
    \bar u_{j+k}^\ell &= \sum_{m=1}^\infty \frac{\Delta x^{m-1}}{m!} 
    \left[
	\sum_{p=0}^m \frac{1-(-1)^p}{2^p}C_m^pk^{m-p} 
    \right]
    \left. \frac{\partial ^{m-1}u}{\partial x^{m-1}}\right|_j^n
    \label{eq:ubar_taylor_l0p} \\
    &=\sum_{m=1}^\infty \frac{\Delta x^{m-1}}{m!} 
    \left[
	\sum_{s=0}^{\left[\frac{m-1}{2}\right]}\frac{1}{4^s} C_m^{2s+1}k^{m-2s-1} 
    \right]
    \left. \frac{\partial ^{m-1}u}{\partial x^{m-1}}\right|_j^n.
    \label{eq:ubar_taylor_l0}
\end{align}
Some special terms for Eq.~(\ref{eq:uVrelation2}) can be deduced by
CAS\footnote{In this work, Mathematica programs are deployed.}
(Computer Algebra System):
\begin{equation}
    \begin{cases}
	\displaystyle
	\bar u_{j}^{n} &\displaystyle =  u_j^n + \frac{1}{24}\Delta x^2 u_{xx} + 
	\frac{1}{1920} \Delta x^4 u_{xxxx} +\cdots\\[3mm]
	\displaystyle
	\bar u_{j-1}^n  &\displaystyle = u_j^n - \Delta x\, u_x + \frac{13}{24} \Delta x^2 u_{xx} -\frac{5}{24}\Delta x^3 u_{xxx}  +\frac{121}{1920}\Delta x^4 u_{xxxx}+\cdots \\[3mm]
	\displaystyle
	\bar u_{j+1}^n  & \displaystyle = u_j^n + \Delta x\, u_x + \frac{13}{24} \Delta x^2 u_{xx} +\frac{5}{24}\Delta x^3 u_{xxx}  +\frac{121}{1920}\Delta x^4 u_{xxxx}+\cdots \\[3mm]
	\displaystyle
	\bar u_{j}^{n+1}& \displaystyle = u_j^n+\Delta t\, u_t + \frac{1}{2}\Delta t^2 u_{tt} + \frac{1}{24} \Delta x^2 u_{xx} + \frac{1}{6}\Delta t^3 u_{ttt} + \frac{1}{24}\Delta t\Delta x^2 u_{xxt} + \cdots   \\[3mm]
	\displaystyle
	\bar u_{j}^{n-1}& \displaystyle = u_j^n-\Delta t\, u_t + \frac{1}{2}\Delta t^2 u_{tt} + \frac{1}{24} \Delta x^2 u_{xx} - \frac{1}{6}\Delta t^3 u_{ttt} -c \frac{1}{24}\Delta t\Delta x^2 u_{xxt} + \cdots
    \end{cases}
    \label{eq:ubar_taylor_many}
\end{equation}




\section{First and Second-Order Schemes}
\label{sec:first-second-order}

\subsection{First-Order Upwind Scheme}
\label{sec:first-order-upwind}

Assuming $a>0$ in Eq.~(\ref{eq:mod-e1}), the first-order upwind
scheme based on nodal value is:
\begin{equation}
    u_j^{n+1} =u_j^n-c(u_j^n-u_{j-1}^n) \qquad 
    c=\frac{a\Delta t}{\Delta x}.
    \label{eq:up1}
\end{equation}
This scheme is first-order in both space and time for point $(j,n)$,
and its modified equation is
\begin{equation}
    u_t+au_x = \frac{a\Delta x}{2}(1-c)u_{xx} - 
    \frac{a\Delta x^2}{6}(1-c)(1-2c)u_{xxx} +
    \frac{a\Delta x^3}{24}(1-c)(1-6c+6c^2)u_{xxxx}+\cdots
    \label{eq:upwind_modified_equ}
\end{equation}
The truncation error of the equation is $O(\Delta x)$.

If the scheme is constructed based on grid average, we should first
integrate Eq.~(\ref{eq:mod-e1}) over interval
$[x_{\jmh},x_{\jph}]\times [t^n,t^{n+1}]$
\begin{equation}
    \bar u_j^{n+1}  = \bar u_j^n -c \left( \hat u_{\jph}-\hat u_{\jmh} \right),
    \label{eq:wave_integral}
\end{equation}
while
\begin{equation}
    \hat u_{\jph} =\frac{1}{\Delta t} \int_{t^n}^{t^{n+1}} u(x_{\jph},t) \diff t
    \label{eq:u_integral}
\end{equation}
is the average value of $u(x,t)$ over a time interval at a fixed
spacial point. Up to the above step, no approximation is introduced, and
all formulae are accurate. Now we would obtain the approximate
expression of $\hat u_{\jph}$ in an explicit scheme. Consider piecewise
constant approximation on each grid $[x_{\jph}, x_{\jph}]$, there are
two values related to point $x_{\jph}$: $\bar u_{j-1}^n$ and
$\bar u_{j+1}^n$, for $a>0$ we take the upwind value:
$\hat u_{\jph}=\bar u_j$ and obtain the same scheme as in
Eq.~(\ref{eq:up1}):
\begin{equation}
    \bar u_j^{n+1}=\bar u_j^n-c(\bar u_j^n-\bar u_{j-1}^n)
    \label{eq:up1V} 
\end{equation}
Utilizing Eq.~(\ref{eq:ubar_taylor_many}) and CAS, modified equation
of Eq.~(\ref{eq:up1V}) is attained:
\begin{equation}
    u_t+au_x = \frac{a\Delta x}{2}(1-c)u_{xx} -
    \frac{a\Delta x^2}{6}(1-c)(1-2c)u_{xxx} +
    \frac{a\Delta x^3}{24}(1-c)(1-6c+6c^2)u_{xxxx}+\cdots
    \label{eq:up1V_modified}
\end{equation}

According to Eq.~(\ref{eq:up1V}) and Eq.~(\ref{eq:up1V_modified}), the
schemes based on nodal values and grid averages accord with each other
at internal grid points. It should be emphasized that numerical
schemes based on grid averages can be constructed by
Eq.~(\ref{eq:ubar_taylor_many}) directly just like constructing FD
schemes with Taylor's expansion. For instance, if time derivatives and
spacial derivatives are both kept to only the first-order partial
derivative, then the expansions of $\bar u^n_{j}$, $\bar u^{n+1}_{j}$, and
$\bar u^n_{j-1}$ give exactly the first-order upwind scheme.

\subsection{Second-Order Central Scheme}
\label{sec:second-order-central}

Apply FTCS Scheme to Eq.~(\ref{eq:mod-e1}), we obtain:
\begin{equation}
    \dfrac{u_j^{n+1}-u_j^n}{\Delta t}+a\dfrac{u_{j+1}^n-u_{j-1}^n}{2\Delta x}=0
    \label{eq:FTCS}
\end{equation}
with its modified equation:
\begin{equation}
    u_t+au_x=-\dfrac{a\Delta x}{2}c\, u_{xx}-\dfrac{a\Delta x^2}{6}(1+2c^2)u_{xxx} -\dfrac{a\Delta x^3}{12}c(2+3c^2)u_{xxxx}+\cdots
    \label{eq:FTCS_modified}
\end{equation}
The Scheme has first-order accuracy in time and second-order accuracy
in space, and its truncation error has a highest order of
two. Meanwhile, we have $\mu_2 = -\dfrac{a}{2}c\Delta x<0$, which
causes instability of the scheme. The instability can be further
demonstrated by Fourier expansion of $\bar u_j^n$, and considering
only one term with wave number $k$, $\bar u_j^n=A_k^n\me^{\mi
kx}$. Substitute this into scheme Eq.~(\ref{eq:FTCS}), the amplification
coefficient is obtained:
\begin{equation}
    G=\frac{A_k^{n+1}}{A_k^n}=1-ic\sin \beta,
    \qquad
    c=\frac{a\Delta t}{\Delta x},\quad \beta=k\Delta x.
    \label{eq:FTCSG} 
\end{equation}
Only $c=0$ satisfies $|G|\leq 1$, which shows that FTCS scheme is
instable.

The FTCS scheme can also be constructed by grid averages as basic
variables. In Eq.~(\ref{eq:ubar_taylor_many}), keep time derivatives
to the first-order and spacial derivatives to the second-order, and
with the first three formulae in Eq.~(\ref{eq:ubar_taylor_many}) and
regard $u_j^n$, $u_x$, and $u_{xx}$ as unknowns, we obtain:
\begin{equation}
    u_x = \frac{\bar u_{j+1}^n-\bar u_{j-1}^n}{2\Delta x} \qquad
    u_{xx} = \frac{\bar u_{j+1}^n-2\bar u_j^n+\bar u_{j-1}^n}{\Delta x^2} \qquad
    u_j^n = \frac{-\bar u_{j+1}^n + 26\bar u_j^n - \bar u_{j-1}^n}{24}.
    \label{eq:du012}
\end{equation}
The expressions above are second-order accurate in space. In the
fourth formula in Eq.~(\ref{eq:ubar_taylor_many}), keep time
derivative to the first-order, and spacial derivative to the 
second-order:
\[
    \bar u_{j}^{n+1} = u_j^n+\Delta t\, u_t + \frac{1}{24} \Delta x^2 u_{xx}.
\]
Solve $u_t$ from the above equation and substitute
Eq.~(\ref{eq:du012}) into expression of $u_t$, we have:
\[
    \Delta t\, u_t =  \bar u_j^{n+1}-\left(u_j^n +\frac{1}{24} \Delta x^2
    u_{xx}\right) = \bar u_j^{n+1}- \bar u_{j}^n.
\]
Substitute these expressions into Eq.~(\ref{eq:mod-e1}), the FTCS
scheme based on grid average is obtained:
\begin{equation}
    \bar u_j^{n+1} = \bar u_j^n - \frac{c}{2} \left(\bar u_{j+1}^n-\bar u_{j-1}^n\right),
    \label{eq:FTCS_grid_avg}
\end{equation}
which is at accord with Eq.~(\ref{eq:FTCS}). It can be further deduced
that the modified equation of Eq.~(\ref{eq:FTCS_grid_avg}) is at
accord with that of Eq.~(\ref{eq:FTCS_modified}).

\subsection{Lax-Wendroff Scheme}
\label{sec:lax-wendroff-scheme}
Taking nodal value as the basic variable in discretization of first-order 
wave equation Eq.~(\ref{eq:mod-e1}), Taylor's expansion can be
applied to $u_j^{n}$ to approximate $u_j^{n+1}$:
\begin{equation}
    u_j^{n+1}=u_j^n+\Delta t\,u_t +\frac{1}{2}\Delta t^2 u_{tt}+O(\Delta t^3).
    \label{eq:LW-T} 
\end{equation}
Transfer the time derivatives into spacial derivatives with
Eq.~(\ref{eq:mod-e1}):
\begin{equation*}
    u_t=-a\, u_x,\hspace{1cm} u_{tt} = a^2 u_{xx},
\end{equation*}
and substitute the above expressions into Eq.~(\ref{eq:LW-T}) we
obtain
\begin{equation}
    u_j^{n+1}=u_j^n-a\Delta t\, u_x +\frac{1}{2}a^2\Delta t^2 u_{xx}+O(\Delta t^3).
    \label{eq:LW-T1} 
\end{equation}
Applying central difference to partial derivatives over $x$, we have
\begin{equation*}
    u_x=\dfrac{u_{j+1}^n-u_{j-1}^n}{2\Delta x} \qquad
    u_{xx}=\dfrac{u_{j+1}^n-2u_j^n+u_{j-1}^n}{\Delta x^2}.
\end{equation*}
Finally the FD scheme of Eq.~(\ref{eq:mod-e1}) is
\begin{equation*}
    \label{eq:LW} \dfrac{u_{j}^{n+1}-u_{j}^n}{\Delta
    t}+a\dfrac{u_{j+1}^n-u_{j-1}^n}{2\Delta x} =\dfrac{1}{2}a^2\Delta
    t\dfrac{u_{j+1}^n-2u_j^n+u_{j-1}^n}{\Delta x^2},
\end{equation*}
which is called Lax-Wendroff Scheme, and it has second-order accuracy
both temporal and spacial. The modified equation of Lax-Wendroff
scheme is
\begin{equation}
    u_t+au_x=-\frac{a\Delta x^2}{6}(1-c^2)u_{xxx} -\frac{a\Delta x^3}{8}c(1-c^2)u_{xxxx}+\cdots
    \label{eq:LW_modified}
\end{equation}
According to Eq.~(\ref{eq:LW_modified}), Lax-Wendroff Scheme is a
dispersion scheme with second-order accuracy. Since the signs of $a$
and $c$ are always the same, $\mu_4=-\frac{a\Delta x^3}{8}c(1-c^2)$ is
always negative given $\abs{c} < 1$, which is to the benefit of
stability of scheme.

The amplification coefficient of the scheme can also be deduced:
\begin{equation}
    G=1-c^2(1-\cos\beta)-ic\sin\beta,\quad\quad
    c=\frac{a\Delta t}{\Delta x},\quad \beta=k\Delta x,
    \label{eq:LWG} 
\end{equation}
\begin{equation}
    \abs{G}=1-4c^2(1-c^2)\sin^4\frac{\beta}{2}.
    \label{eq:LWGA} 
\end{equation}
The stability condition is
\begin{equation}
    \abs{c}\leq 1,\quad\mbox{which means}\quad
    \frac{\abs{a}\Delta t}{\Delta x}\leq 1.
    \label{eq:LWGA1} 
\end{equation}

The above Lax-Wendroff's process can also be applied to case with grid
average as basic variable, and construct the corresponding scheme.
According to the fourth one in Eq.~(\ref{eq:ubar_taylor_many}):
\[
    \bar u_{j}^{n+1}  = u_j^n+\Delta t\, u_t + \frac{1}{2}\Delta t^2
    u_{tt} + \frac{1}{24} \Delta x^2 u_{xx}
\]
and substitute Eq.~(\ref{eq:du012}) into it:
\[
    \bar u_j^{n+1}  = \frac{-\bar u_{j+1}^n + 26\bar u_j^n - \bar u_{j-1}^n}{24}-
    \frac{c}{2}\left(\bar u_{j+1}^n-\bar u_{j-1}^n\right) +
    \frac{12c^2+1}{24}\left(\bar u_{j+1}^n-2\bar u_j^n+\bar
    u_{j-1}^n\right),
\]
which is actually
\begin{equation}
    \bar u_j^{n+1} = \bar u_j^n -\frac{c}{2}\left(\bar u_{j+1}^n-\bar u_{j-1}^n\right) +
    \frac{c^2}{2}\left(\bar u_{j+1}^n-2\bar u_j^n+\bar u_{j-1}^n\right).
    \label{eq:LWav}
\end{equation}
Eq.~(\ref{eq:LWav}) is consistent with Eq.~(\ref{eq:LW}). The
modified equation can also be deduced:
\begin{equation*}
    u_t+au_x=-\frac{a\Delta x^2}{6}(1-c^2)u_{xxx} -\frac{a\Delta x^3}{8}c(1-c^2)u_{xxxx}+\cdots
\end{equation*}
which is the same as Eq.~(\ref{eq:LW_modified}).




\section{Higher Order Schemes}
\label{sec:higher-order-schemes}

QUICK scheme is taked as an example of constructing high order scheme in this
section.

\subsection{QUICK Scheme Based on Grid Average}
\label{sec:quick-scheme-based}

The general scheme for the wave equation $u_t+a u_x=0$ with
conservative semi-discretization and with grid average as basic variable:
\begin{equation}
    \frac{d}{dt}\bar u_j = -\frac{1}{\Delta x}(f_{\jph}-f_{\jmh})=\sum_{k=-J}^{J-1}\alpha_k\left(\bar u_{j+k+1}-\bar u_{j+k}\right).
    \label{eq:linearsch}
\end{equation}
It is a scheme of $2J+1$ grid elements, $\bar u_j$ means grid
average on $[x_{\jmh}, x_\jph]$, and $f_\jph$ is numerical flux in the
form of $f_{\jph} = \hat f(\bar u_{j-J+1},\ldots,\bar u_{j+J})$ which
is a function of $2J$ variables.
We cannot calculate Taylor's expansion on $\bar u_j$ since it is a
grid average instead of nodal value. Denote the antiderivative of
$u(x,t)$ as $V(x,t)$, and $V_\jph = V(x_\jph,t)$, then according to
definition of grid average:
\begin{equation}
    \bar u_{j+k}=\frac{V_{j+k+1/2}-V_{j+k-1/2}}{\Delta x}.
    \label{eq:ujkV}
\end{equation}
Substituting the above equation to Eq.~(\ref{eq:linearsch}), we have 
\begin{equation}
    \frac{d}{dt}\left(\frac{V_{\jph}-V_{\jmh}}{\Delta x}\right)=\sum_{k=-J}^{J-1}\alpha_k\frac{V_{j+k+\frac{3}{2}}-2V_{j+k+1/2}+V_{j+k-\frac{1}{2}}}{\Delta x}.
    \label{eq:linearschV}
\end{equation}
The modified equation of Eq.~(\ref{eq:linearsch}) can be deduced by
Eq.~(\ref{eq:linearschV}) based on Taylor's expansion of $V_{j+k+\frac 12}$.

\subsubsection{Construction of QUICK Scheme}
\label{sec:construct-quick}

We take Leonard's scheme \cite{leonard1979stable}
as example of constructing high order schemes, which is a third-order scheme
discretizing the convection term with upwind scheme and directly constructing
flux across cell interface and is called (QUICK) scheme.
The primary scheme is constructed with nodal value
and fails to achieve third-order accuracy due to imbalance of equation.
A more reasonable method is constructing interpolation in terms of grid 
averages,
which leads to a new scheme different from the previous one.

Integrate Eq.~(\ref{eq:conservative-convection}) over 
interval $[x_\jmh, x_\jph]$ containing point $x_j$, we have
\begin{equation}
    \frac{d \overline{u}_j}{d t}+\frac{1}{\Delta x}\left(f_{\jph}-f_{\jmh}\right)=0,
    \label{wave-conix}
\end{equation}
while $\bar u_j$ is grid average of $u(x,t)$ on interval $(x_\jmh,x_\jph)$
defined by Eq.~(\ref{eq:def_grid_avg}). Eq.~(\ref{wave-conix}) is an accurate
expression, but the interface flux $f_\jph$ should be approximated by $\bar
u_j$. 

\begin{figure}[htb]
    \begin{center}
	\unitlength=1mm
	\begin{picture}(120,25)
	    \thicklines
	    \put(20,0){\line(1,0){80}}
	    \put(20,12){\color{cyan}\line(1,0){60}}
	    \put(40,19){\color{green}\line(1,0){60}}
	    \put(50, 0.2){\line(1,0){20}}
	    \put(50,-0.2){\line(1,0){20}}
	    \thinlines
	    \multiput(20,0)(20,0){5}{\color{blue}\line(0,1){25}}
	    \multiput(30,0)(20,0){4}{\cb{$\bullet$}}\put(17,0){\cb{$S_0$}}
	    \multiput(30,12)(20,0){3}{\cb{\color{cyan}$\bullet$}}\put(17,12){\cb{\color{cyan}$S_\ell$}}
	    \multiput(50,19)(20,0){3}{\cb{\color{green}$\bullet$}}\put(103,19){\color{green}\cb{$S_r$}}
	    \put( 30,4){\cb{$x_{j-1}$}}
	    \put( 50,4){\cb{$x_j$}}
	    \put( 70,4){\cb{$x_{j+1}$}}
	    \put( 90,4){\cb{$x_{j+2}$}}
	    \put( 60,-3){\cb{\color{red}$x_{\jph}$}}
	    \thicklines
	    \put(60,0){\color{red}\line(0,1){25}}
	\end{picture}
    \end{center}
    \caption{The sets on which interpolation of QUICK is operated: $S_\ell=\{\bar u_{j-1}, \bar u_j,\bar u_{j+1}\}$, $S_r = \{\bar u_j, \bar u_{j+1}, \bar u_{j+2} \}$, and $S_0=\{\bar u_{j-1}, \bar u_j,\bar u_{j+1}, \bar u_{j+2} \}$.}
    \label{fig:QUICK-S}
\end{figure}

In order to obtain value of a particular function on point $x_\jph$, a 
second-order or third-order polynomial interpolant 
should be constructed on interval
$[x_j,x_\jph]$. As shown in Fig.~\ref{fig:QUICK-S}, $S_\ell$ is 
utilized to
construct second-order interpolant with $a>0$, $S_r$ is utilized to
construct second-order interpolant with $a\leq0$, and $S_0$ is utilized to
construct third-order interpolant. The second-order interpolant in QUICK scheme
can be expressed as (assuming that 
the origin of coordinate is at interface $x_\jph = 0$):
\begin{equation*}
    p(x) = C_0 + C_1 x + C_2 x^2.
\end{equation*}
Obviously $C_0$ is the value on interface as $x_\jph=0$. Given average on
interval $[x_a, x_b]$, we have
\begin{equation*}
    \bar u  =C_0+\frac{1}{2}C_1(x_a+x_b)+\frac{1}{3}C_2(x_a^2+x_ax_b+x_b^2).
\end{equation*}
While $a>0$, we construct interpolant on set $S_\ell$, and obtain the equations
on coefficients of interpolant:
\[
    \Matrix{\begin{array}{crr}
	1 & -\frac{3}{2}\Delta x & \frac{7}{3}\Delta x^2 \\
	1 & -\frac{1}{2}\Delta x & \frac{1}{3}\Delta x^2 \\
	1 &  \frac{1}{2}\Delta x & \frac{1}{3}\Delta x^2 \\
    \end{array}}
    \Matrix{\begin{array}{c} C_0 \\ C_1 \\ C_2 \\ \end{array}} =
    \Matrix{\begin{array}{l} \bar u_{j-1} \\ \bar u_j \\ \bar u_{j+1} \\ \end{array}},
\]
As the coefficient matrix possesses a determinant of $2\Delta x^3\neq 0 $,
it is reversible, and the equations has unique solution
\[
    \left\{\begin{array}{l}
	\displaystyle
	C_0=\frac{1}{6}(-\bar u_{j-1}+5\bar u_j+2\bar u_{j+1}),\\[3mm]
	\displaystyle
	C_1=\frac{1}{\Delta x}(\bar u_{j+1}-\bar u_j),\\[3mm]
	\displaystyle
	C_2=\frac{1}{2\Delta x^2}(\bar u_{j-1}-2\bar u_j+\bar u_{j+1}).\\
    \end{array}\right.
\]
Now we have
\[u_{j+1/2}=\frac{1}{6}(-\bar u_{j-1}+5\bar u_j+2\bar u_{j+1})=\frac{1}{2}(\bar
u_j+\bar u_{j+1})-\frac{1}{6}(\bar u_{j-1}-2\bar u_j+\bar u_{j+1});\]
similarly, when $a<0$ we can deduce the corresponding expression on set
$S_r$. The final result is
\begin{equation}
    u_{j+1/2}=\left\{\begin{array}{ll}
	\displaystyle
	\frac{1}{2}(\bar u_j+\bar u_{j+1})-\frac{1}{6}(\bar u_{j-1}-2\bar u_j+\bar u_{j+1}),\hspace{1cm} & a>0;\\[3mm]
	\displaystyle
	\frac{1}{2}(\bar u_j+\bar u_{j+1})-\frac{1}{6}(\bar u_j-2\bar u_{j+1}+\bar u_{j+2}),\hspace{1cm} & a\le 0.\\
    \end{array}\right.
    \label{eq:QUICK-flux}
\end{equation}
Eq.~(\ref{eq:QUICK-flux}) has similar form but different coefficients compared
with literature \cite{leonard1979stable}: it is reported that there is $1/8$
instead of $1/6$.
Eq.~(\ref{eq:QUICK-flux}) can be rewritten as fourth-order central scheme with
correction term
\begin{equation}
    u_{j+1/2}=\left\{\begin{array}{ll}
	\displaystyle
	\frac{7}{12}(\bar u_j+\bar u_{j+1})-\frac{1}{12}(\bar u_{j-1}+\bar u_{j+2}){\color{red}+\frac{1}{12}\left(\bar u_{j+2}-3\bar u_{j+1}+3\bar u_{j}-\bar u_{j-1} \right)}     ,\hspace{2mm} & a>0;\\[3mm]
	\displaystyle
	\frac{7}{12}(\bar u_j+\bar u_{j+1})-\frac{1}{12}(\bar u_{j-1}+\bar u_{j+2}){\color{red}-\frac{1}{12}\left(\bar u_{j+2}-3\bar u_{j+1}+3\bar u_{j}-\bar u_{j-1} \right)}     ,\hspace{2mm} & a\le 0.\\
    \end{array}\right.
    \label{eq:QUICK-flux-2}
\end{equation}
Denoting $\kappa = \sgn a$ and defining $\bar f_j = f(\bar u_j) = a\bar u_j$, we
have the flux form of Eq.~(\ref{eq:QUICK-flux-2})
\begin{equation}
    f_{j+1/2}=
    \frac{7}{12}(\bar f_j+\bar f_{j+1})-\frac{1}{12}(\bar f_{j-1}+\bar
    f_{j+2}){\color{red}+\frac{1}{12}\kappa a\left(\bar u_{j+2}-3\bar u_{j+1}+3\bar
    u_{j}-\bar u_{j-1} \right)}.
    \label{eq:QUICK-flux-3}
\end{equation}
The correction term can be analyzed with antiderivative of $u(x,t)$; rewrite
correction terms with Eq.~(\ref{eq:def_grid_avg}): 
\begin{equation*}
    \begin{array}{ll}
	\bar u_{j+2}-3\bar u_{j+1}+3\bar u_{j}-\bar u_{j-1}&
	\displaystyle =\frac{1}{\Delta x}
	\left(V_{j+5/2}-4V_{j+3/2}+6V_{j+1/2}-4V_{j-1/2}+V_{j-3/2}\right) \\[3mm]
	&\displaystyle
	=\frac{\partial^4 V}{\partial x^4}\Delta x^3+\frac{1}{6}\frac{\partial^6V}{\partial x^6}\Delta x^5 + \cdots \\[3mm]
	&\displaystyle
	=\frac{\partial^3 u}{\partial x^3}\Delta x^3+\frac{1}{6}\frac{\partial^5u}{\partial x^5}\Delta x^5 + \cdots \\
    \end{array}
\end{equation*}
The partial derivatives in the above equation are at point $x_\jph$. In
Eq.~(\ref{eq:QUICK-flux-2}) by definition of $\kappa$, $\kappa=1$ corresponds to
case of $a>0$, $\kappa=-1$ corresponds to case of $a<0$, and the scheme becomes
a fourth-order one when $\kappa=0$.

Substitute the flux given by Eq.~(\ref{eq:QUICK-flux-3}) into
Eq~(\ref{wave-conix}), and rewrite the scheme into form of
Eq.~(\ref{eq:linearsch}); in this case, we have $J = 2$ and
\begin{equation}
    \left\{\begin{array}{ll}
	\alpha_{-2}&\displaystyle =\ \ \frac{1}{12\Delta x}(1+\kappa)a,\\[3mm]
	\alpha_{-1}&\displaystyle = -\frac{1}{12\Delta x}(7+3\kappa)a,\\[3mm]
	\alpha_{0}&\displaystyle = -\frac{1}{12\Delta x}(7-3\kappa)a,\\[3mm]
	\alpha_{1}&\displaystyle =\ \ \frac{1}{12\Delta x}(1-\kappa)a.\\
    \end{array}\right.
    \label{eq:coeff-of-alpha-in-quick}
\end{equation}
The coefficients given above can be applied to Eq.~(\ref{eq:linearschV})
and the modified equation is obtained:
\begin{equation}
    u_t+au_x =-\frac{1}{12}\kappa \Delta x^3 u_{xxxx}+\frac{1}{30}a \Delta x^4 u_{xxxxx} -\frac{1}{72} \kappa a \Delta x^5 u_{xxxxxx} +\cdots
    \label{eq:modified-quick}
\end{equation}
The leading term of truncation error in the above equation is a fourth-order
one, which implies that the scheme is third-order accurate.


\subsubsection{The Properties of Interpolation Coefficient with Grid Average as Basic Variable}
\label{sec:prop-coeff-grid-avg}

A third-order polynomial can be constructed given set $S_0$ shown in
Fig.~\ref{fig:QUICK-S}:
\begin{equation*}
    p(x) = C_0 + C_1 x + C_2 x^2 + C_3 x^3.
\end{equation*}
Setting point $x_\jph$ as the origin of coordinate, the equations of
coefficients to be determined are:
\begin{equation*}
    \Matrix{\begin{array}{crrr}
	1 & -\frac{3}{2}\Delta x & \frac{7}{3}\Delta x^2 & -\frac{15}{4}\Delta x^3 \\
	1 & -\frac{1}{2}\Delta x & \frac{1}{3}\Delta x^2 & -\frac{1}{4}\Delta x^3 \\
	1 & \frac{1}{2}\Delta x & \frac{1}{3}\Delta x^2 & \frac{1}{4}\Delta x^3 \\
	1 & \frac{3}{2}\Delta x & \frac{7}{3}\Delta x^2 & \frac{15}{4}\Delta x^3 \\
    \end{array}}
    =\Matrix{\begin{array}{l}
	\bar u_{j-1}\\ \bar u_{j} \\ \bar u_{j+1} \\ \bar u_{j+2} \\
    \end{array}}.
\end{equation*}
The determinant of the coefficient matrix is $D = 12\Delta x^6$, so there exists
unique solution
\begin{equation*}
    \left\{\begin{array}{l}
	\displaystyle
	C_0 = \frac{1}{12}\left[7(\bar u_j+\bar u_{j+1})-(\bar u_{j-1}+\bar u_{j+2})\right],\\[3mm]
	\displaystyle
	C_1= \frac{1}{12\Delta x}\left[15(\bar u_{j+1}-\bar u_{j-1})-(\bar u_{j+2}-\bar u_{j-1})\right],\\[3mm]
	\displaystyle
	C_2 = -\frac{1}{4\Delta x^2}\left[(\bar u_j+\bar u_{j+1})-(\bar u_{j-1}+\bar u_{j+2})\right],\\[3mm]
	\displaystyle
	C_3= -\frac{1}{6\Delta x^3}\left[3(\bar u_{j+1}-\bar u_{j-1})-(\bar u_{j+2}-\bar u_{j-1})\right].\\
    \end{array}\right.
\end{equation*}
The interpolation coefficients based on set 
$S_\ell$ are
\begin{equation*}
    \left\{\begin{array}{l}
	\displaystyle
	C_0 = \frac{1}{6}(-\bar u_{j-1}+5\bar u_j+2\bar u_{j+1}),\\[3mm]
	\displaystyle
	C_1= \frac{1}{\Delta x}(\bar u_{j+1}-\bar u_{j}),\\[3mm]
	\displaystyle
	C_2 = \frac{1}{2\Delta x^2}(\bar u_{j-1}-2\bar u_j+\bar u_{j+1});\\[3mm]
    \end{array}\right.
\end{equation*}
The interpolation coefficients based on set $S_r$ are
\begin{equation*}
    \left\{\begin{array}{l}
	\displaystyle
	C_0 = \frac{1}{6}(2\bar u_{j}+5\bar u_{j+1}-\bar u_{j+2}),\\[3mm]
	\displaystyle
	C_1= \frac{1}{\Delta x}(\bar u_{j+1}-\bar u_{j}),\\[3mm]
	\displaystyle
	C_2 = \frac{1}{2\Delta x^2}(\bar u_{j}-2\bar u_{j+1}+\bar u_{j+2}).\\[3mm]
    \end{array}\right.
\end{equation*}
Let superscript ``$(3)$'' and ``$(4)$'' denote the coefficients of the second
and third-order interpolation polynomial:
\begin{equation}
    \left\{\begin{array}{l}
	\displaystyle
	C_0^{(3)}=C_0^{(4)}+\frac{1}{2}\kappa C_3^{(4)}\Delta x^3, \\[3mm]
	\displaystyle
	C_1^{(3)}=C_1^{(4)}+\frac{1}{2}\abs{\kappa} C_3^{(4)}\Delta x^2, \\[3mm]
	\displaystyle
	C_2^{(3)}=C_2^{(4)}-\frac{3}{2}\kappa C_3^{(4)}\Delta x, \\
    \end{array}\right.
    \label{eq:ave-C-34}
\end{equation}
$\kappa = 1$ for $S_\ell$ and $\kappa = -1$ for $S_r$. It is worthwhile pointing
out that $C_1^{(3)}$ is not dependent on parameter $\kappa$ which means that it
is independent of deflection of base set. The expression $|\kappa|$ is used to
ensure that the whole formula corresponds to base set $S_0$ when $\kappa=0$. The
mathematical definition of $\kappa$ is $\kappa = \mathrm{sgn} a = a/|a|$, but
in order to avoid situation where zero is divided, we can redefine $\kappa$ as
\begin{equation*}
    \kappa =\frac{a}{Q(a)},\hspace{1cm}Q(x) = \left\{\begin{array}{ll} \abs{x}, & \abs{x}\ge \varepsilon \\
	\varepsilon, & \abs{x} < \varepsilon \\
    \end{array}\right.\hspace{5mm}\mbox{or}\hspace{5mm}
    \left\{\begin{array}{ll} \abs{x}, & \abs{x}\ge \varepsilon \\
	\frac{x^2+\varepsilon^2}{2\varepsilon}, & \abs{x} < \varepsilon \\ \end{array}\right.
\end{equation*}

The property of coefficients given by Eq.~(\ref{eq:ave-C-34}) can be
generalized.
Given averages on $m$ intervals:
\begin{equation}
    \frac{1}{x_{j+1}-x_j}\int_{x_j}^{x_{j+1}}u(x)dx=\bar
    u_j,\hspace{1cm}j=1,2,\cdots,m,
    \label{eq:ave-uj-m}
\end{equation}
a $(m-1)$th-order polynomial can be constructed as there are $m$ constrains:
\begin{equation}
    p_{m-1}(x) = C_0^{(m)}+C_1^{(m)}x + \cdots +C_{m-1}^{(m)}x^{m-1}.
    \label{eq:p-m1-x}
\end{equation}
Substituting $u(x) = p_{m-1}(x)$ into Eq.~(\ref{eq:ave-uj-m}), we obtain
\begin{equation}
    C_0^{(m)}+\frac{1}{2}\frac{x_{j+1}^2-x_j^2}{x_{j+1}-x_j}C_1^{(m)}+\frac{1}{3}\frac{x_{j+1}^3-x_j^3}{x_{j+1}-x_j}C_2^{(m)}+\cdots+\frac{1}{m}\frac{x_{j+1}^m-x_j^m}{x_{j+1}-x_j}C_{m-1}^{(m)} =\bar u_j.
    \label{eq:ave-uj-C}
\end{equation}
While $j=1,2,\ldots,m$, a linear equation about 
$ \bm C^{(m)}=(C_0^{(m)},C_1^{(m)},\cdots,C_{m-1}^{(m)})$ is 
\begin{equation}
    W^{(m)}\bm C^{(m)}=\bm R^{(m)},
    \label{eq:WC=R}
\end{equation}
and the coefficient matrix and the left hand side term is
\begin{equation*}
    W^{(m)}\equiv\Matrix{\begin{array}{ccccc}
	1& \frac{1}{2}\frac{x_{2}^2-x_1^2}{x_{2}-x_1} &\frac{1}{3}\frac{x_{2}^3-x_1^3}{x_{2}-x_1} & \cdots & \frac{1}{m}\frac{x_{2}^m-x_1^m}{x_{2}-x_1} \\[3mm]
	1& \frac{1}{2}\frac{x_{3}^2-x_2^2}{x_{3}-x_2} &\frac{1}{3}\frac{x_{3}^3-x_2^3}{x_{3}-x_2} & \cdots & \frac{1}{m}\frac{x_{3}^m-x_2^m}{x_{3}-x_2} \\
	\cdots &\cdots &\cdots &\cdots &\cdots \\
	1& \frac{1}{2}\frac{x_{m+1}^2-x_m^2}{x_{m+1}-x_m} &\frac{1}{3}\frac{x_{m+1}^3-x_m^3}{x_{m+1}-x_m} & \cdots & \frac{1}{m}\frac{x_{m+1}^m-x_m^m}{x_{m+1}-x_m} \\
    \end{array}},\hspace{5mm}
    \bm R^{(m)}=\Matrix{\begin{array}{c}\bar u_1\\ \bar u_2\\ \vdots\\ \bar u_m\\ \end{array}}.
\end{equation*}
The determinant of coefficient matrix $W^{(m)}$ is
\begin{equation*}
    \abs{W^{(m)}}=\frac{1}{\displaystyle m!\prod_{k=1}^m(x_{k+1}-x_k)}\abs{
	\begin{array}{ccccc}
	    x_2-x_1 & x_2^2-x_1^2 & x_2^3-x_1^3 & \cdots & x_2^m-x_1^m \\[3mm]
	    x_3-x_2 & x_3^2-x_2^2 & x_3^3-x_2^3 & \cdots & x_3^m-x_2^m \\
	    \cdots &\cdots &\cdots &\cdots &\cdots \\
	    x_{m+1}-x_m & x_{m+1}^2-x_m^2 & x_{m+1}^3-x_m^3 & \cdots & x_{m+1}^m-x_m^m
	\end{array}
    },
\end{equation*}
and it can be deduced that\footnote{The proof is as following. Adding one row
    onto the determinant as the first row and one column as the first column. The
    added row is $(1, x_1, x_1^2, x_1^3, \ldots, x_1^m)$, and the added column
    consists of zeros except that first element is $1$. The new determinant is a
    $(m+1)$th-order one; adding each row onto the next, we attain a standard
$(m+1)$th-order Vandermonde determinant.}
\begin{equation}
    \abs{\begin{array}{ccccc}
	x_2-x_1 & x_2^2-x_1^2 & x_2^3-x_1^3 & \cdots & x_2^m-x_1^m \\[3mm]
	x_3-x_2 & x_3^2-x_2^2 & x_3^3-x_2^3 & \cdots & x_3^m-x_2^m \\
	\cdots &\cdots &\cdots &\cdots &\cdots \\
	x_{m+1}-x_m & x_{m+1}^2-x_m^2 & x_{m+1}^3-x_m^3 & \cdots & x_{m+1}^m-x_m^m 
    \end{array}
}=\prod_{i_1<i_2}(x_{i_2}-x_{i_1}).
\label{eq:Vande2}
\end{equation}
The total number of terms in the RHS is $C_{m+1}^2 = \frac{m(m+1)}{2}$. Given
$\forall k, x_k \neq x_{k+1}$, we have
\begin{equation*}
    \abs{W^{(m)}} = \frac{1}{m!}\prod_{i_1+1<i_2}(x_{i_2}-x_{i_1}),
\end{equation*}
so the $m$th-order equations about $(C_0, C_1, \ldots, C_{m-1}$ has unique
solution. Then we consider two sets of equations about $\bm C^{(n)}$ and
$\bm C^{(n-1)}$ respectively:
\begin{equation}
    W^{(n)}\bm C^{(n)}=\bm R^{(n)},\hspace{1cm}W^{(n-1)}\bm C^{(n-1)}=\bm R^{(n-1)}.
    \label{eq:C-n-C-n1}
\end{equation}
%%%%% TBD %%%%%
In the equation above, matrix $W^{(n-1)}$ is submatrix of $W^{(n)}$ with the
$n$th row and the $n$th column removed, and column vector $\bm R^{(n-1)}$
consisting the first $(n-1)$ components of column vector $\bm R^{(n)}$.
Denoting the first $(n-1)$ components of $\bm C^{(n)}$ as $\bm C_s^{(n-1)}$, the
first equation of Eq.~(\ref{eq:C-n-C-n1}) can be rewriten in the block matrix
form:
\begin{equation}
    \Matrix{\begin{array}{cc}
W^{(n-1)} & \bm x^{(n)}_s \\ \times & \frac{1}{n}\frac{x_{n+1}^n-x_1^n}{x_{n+1}-x_n} \\
\end{array}}
\Matrix{\begin{array}{c}\bm C^{(n)}_s \\ C^{(n)}_{n-1} \\ \end{array}} =
\Matrix{\begin{array}{c}\bm R^{(n-1)} \\ \bar u_{n} \\ \end{array}},
    \label{eq:C-n-C-n1-block}
\end{equation}
and $\bm x_s$ is a $(n-1)$-dimentional column vector of the last column of matrix
$W^{(n)}$ without the last element:
\[
\a x_s^{(n)}=\left(\frac{1}{n}\frac{x_2^n-x_1^n}{x_2-x_1},\frac{1}{n}\frac{x_3^n-x_2^n}{x_3-x_2},\cdots,\frac{1}{n}\frac{x_{n}^n-x_{n-1}^n}{x_{n}-x_{n-1}}\right)^T.
\]
According to Eq.~(\ref{eq:C-n-C-n1-block}), we have
\begin{equation}
    W^{(n-1)}\bm C^{(n)}_s + C^{(n)}_{n-1}\bm x_s^{(n)}=\bm R^{(n-1)}.
    \label{eq:C-n-C-n1-sub}
\end{equation}
Denoting the inverse matrix of $W^{(n-1)}$ as $\Lambda^{(n-1)}$ and combining
Eq.~(\ref{C-n-C-n1-sub}) and the first equation in Eq.~(\ref{eq:C-n-C-n1}), we
have
\begin{equation}
    \bm C^{(n-1)}=\Lambda^{(n-1)}\bm R^{(n-1)}=\bm C^{(n)}_s + C^{(n)}_{n-1}\Lambda^{(n-1)}\bm x_s^{(n)}.
    \label{eq:C-n-C-n1-gamma}
\end{equation}
An observation of Eq.~(\ref{C-n-C-n1-gamma}) at $n=3$ is
\[
    \Matrix{\begin{array}{c}C_0^{(2)}\\ C_1^{(2)}\\ \end{array}}=
    \Matrix{\begin{array}{c}C_0^{(3)}\\ C_1^{(3)}\\ \end{array}}
    +C_2^{(3)}\Matrix{\begin{array}{c} -\frac{1}{3}(x_1x_2+x_1x_3+x_2x_3)\\ \frac{2}{3}(x_1+x_2+x_3) \\ \end{array}};
\]
and the case of $n=4$:
\[
    \Matrix{\begin{array}{c}C_0^{(3)}\\ C_1^{(3)}\\ C_2^{(3)}\\ \end{array}}=
    \Matrix{\begin{array}{c}C_0^{(4)}\\ C_1^{(4)}\\ C_2^{(4)}\\ \end{array}}
    +C_3^{(4)}\Matrix{\begin{array}{c} \frac{1}{4}\sum_{i_1<i_2<i_3} x_{i_1}x_{i_2}x_{i_3} \\ -\frac{1}{2}\sum_{i_1<i_2} x_{i_1}x_{i_2} \\ \frac{3}{4}(x_1+x_2+x_3+x_4) \\ \end{array}}.
\]
One conjecture can be made that $\forall n$, $\Lambda^{(n-1)}\bm x_s^{(n)}$ is
$(n-1)$-dimentional column vector:
\begin{equation}
    \Lambda^{(n-1)}\bm x_s^{(n)}=\Matrix{\begin{array}{r}
	\displaystyle
	-\frac{(-1)^{n-1}}{n}\sum_{i_1<i_2<\cdots<i_{n-1}}^n x_{i_1}x_{i_2}\cdots x_{i_{n-1}} \\
	\vdots\hspace{1cm} \\
	\displaystyle
	-\frac{n-2}{n}\sum_{i_1<i_2}^n x_{i_1}x_{i_2} \\[5mm]
	\displaystyle
	\frac{n-1}{n}\sum_{i=1}^n x_i \\
    \end{array}}\equiv\bm s.
    \label{eq:C-n-C-n1-s}
\end{equation}
The proof of Eq.~(\ref{eq:C-n-C-n1-s}) is equivalent to proving
$W^{(n-1)}\bm s=\bm x_s^{(n)}$, and the later formula is proved rigorously in
Appedix~\ref{sec:proof}.


\subsection{QUICK Scheme Based Nodal Value}

If we regard $\bar u_j$ as nodal value $u_j$, which is what the primary QUICK
scheme actually assumes \cite{leonard1979stable}, the semi-discrete scheme is
\begin{equation}
    \frac{d u_j}{d t}+\frac{1}{\Delta x}\left(f_{\jph}-f_{\jmh}\right)=0,
    \label{eq:QUICK-duj-dt}
\end{equation}
with $f_\jph$ being the flux across interface. Flux $f$ is approximated by
quadratic polynomials. According to the sign of $a_\jph = \partial f/\partial
u$, the quadratic providing $f_\jph$ is constructed by $\{ x_{j-1}, x_j,
x_{j+1}
\}$ or $\{ x_j, x_{j+1}, x_{j+2} \}$ to satisfy upwind
property. Flux $f_\jph$ at interface $x_\jph$ is
\begin{equation*}
    f_{\jph}=\left\{\begin{array}{ll}
	\displaystyle
	\frac{1}{2}(f_j+f_{j+1})-\frac{1}{8}(f_{j-1}-2f_j+f_{j+1}),\hspace{1cm} & a_{\jph}>0;\\[3mm]
	\displaystyle
	\frac{1}{2}(f_j+f_{j+1})-\frac{1}{8}(f_j-2f_{j+1}+f_{j+2}),\hspace{1cm} & a_{\jph}\le 0.\\
    \end{array}\right.
\end{equation*}
The above equation differs from Eq.~(\ref{eq:QUICK-flux}), but still seems hold
third-order accuracy in terms of approximation of flux. Further analysis on wave
equation with $f=au$ is done
by Taylor's expansion directly to nodal values, and it shows that the scheme
actually has only second-order accuracy. We rewrite the flux term into
fourth-order central scheme with correction term (the red-colored part):
\begin{equation}
    f_{\jph}=\frac{9}{16}(f_j+f_{j+1})-\frac{1}{16}(f_{j-1}+f_{j+2}){\color{red}+\frac{1}{16}\abs{a}(u_{j+2}-3u_{j+1}+3u_{j}-u_{j-1})},
    \label{eq:QUICK-flux-nodal}
\end{equation}
and we can deduce that:
\[
    u_{j+2}-3u_{j+1}+3u_{j}-u_{j-1}=\frac{\partial^3 u}{\partial x^3}\Delta x^3+\frac{1}{8}\frac{\partial^5 u}{\partial x^5}\Delta x^5+\cdots
\]
The coefficient in front of truncation error term is $1/8$ instead of $1/6$ in
the case of grid average. The modified equation of
scheme~(\ref{eq:QUICK-duj-dt}) is 
\begin{equation}
    u_t+au_x={\color{red}-\frac{1}{24}a\Delta x^2 u_{xxx}}+\frac{1}{16}\abs{a}\Delta x^3 u_{xxxx}+\frac{11}{380}a\Delta x^4 u_{xxxxx} +\cdots
    \label{eq:mod-quick-nodal}
\end{equation}
The leading term of truncation error of Eq.~(\ref{eq:mod-quick-nodal}) is
third-order, which means that the scheme has only second-order accuracy instead
of what is expected. The reason why third-order accuracy is not achieved is that
grid average and nodal value are not strictly distinguished: the LHS of
Eq.~(\ref{eq:QUICK-duj-dt}) should not be nodal values. The disparity on
coefficients of correction term mentioned above is the cause of loss of
accuracy.

\begin{figure}[htb]
    \begin{center}
	\unitlength=1mm
	\begin{picture}(120,25)
	    \thicklines
	    \put(10,0){\line(1,0){100}}
	    \put(40, 0.2){\color{cyan}\line(1,0){20}}
	    \put(40,-0.2){\color{cyan}\line(1,0){20}}
	    \put(60, 0.2){\color{green}\line(1,0){20}}
	    \put(60,-0.2){\color{green}\line(1,0){20}}
	    \put(50, 2.2){\line(1,0){20}}
	    \put(50, 2.0){\line(1,0){20}}
	    \put(50, 1.8){\line(1,0){20}}
	    \put(0,0){\color{cyan}\qbezier(40,8)(50,15)(60,12)}
	    \put(0,0){\color{green}\qbezier(60,12)(70,6)(80,7)}
	    \multiput(51,0)(0,1){13}{\color{cyan}\cb{$\cdot$}}
	    \multiput(52,0.5)(0,1){12}{\color{cyan}\cb{$\cdot$}}
	    \multiput(53,0)(0,1){13}{\color{cyan}\cb{$\cdot$}}
	    \multiput(54,0.5)(0,1){13}{\color{cyan}\cb{$\cdot$}}
	    \multiput(55,0)(0,1){13}{\color{cyan}\cb{$\cdot$}}
	    \multiput(56,0.5)(0,1){12}{\color{cyan}\cb{$\cdot$}}
	    \multiput(57,0)(0,1){13}{\color{cyan}\cb{$\cdot$}}
	    \multiput(58,0.5)(0,1){12}{\color{cyan}\cb{$\cdot$}}
	    \multiput(59,0)(0,1){12}{\color{cyan}\cb{$\cdot$}}
	    \multiput(60,0.5)(0,1){12}{\color{green}\cb{$\cdot$}}
	    \multiput(61,0)(0,1){11}{\color{green}\cb{$\cdot$}}
	    \multiput(62,0.5)(0,1){11}{\color{green}\cb{$\cdot$}}
	    \multiput(63,0)(0,1){11}{\color{green}\cb{$\cdot$}}
	    \multiput(64,0.5)(0,1){10}{\color{green}\cb{$\cdot$}}
	    \multiput(65,0)(0,1){10}{\color{green}\cb{$\cdot$}}
	    \multiput(66,0.5)(0,1){9}{\color{green}\cb{$\cdot$}}
	    \multiput(67,0)(0,1){9}{\color{green}\cb{$\cdot$}}
	    \multiput(68,0.5)(0,1){8}{\color{green}\cb{$\cdot$}}
	    \multiput(69,0)(0,1){8}{\color{green}\cb{$\cdot$}}
	    \put(55,6){\cb{$I_{j^-}^+$}}
	    \put(65,6){\cb{$I_{j^+}^+$}}
	    \put(60,-3){\cb{$x_j$}}
	    \thinlines
	    \put(10,15){\color{cyan}\line(1,0){60}}
	    \put(30,19){\color{green}\line(1,0){60}}
	    \multiput(10,0)(20,0){6}{\color{blue}\line(0,1){25}}
	    \multiput(20,0)(20,0){5}{\cb{$\bullet$}}
	    \multiput(20,15)(20,0){3}{\cb{\color{cyan}$\bullet$}}\put(7,15){\cb{\color{cyan}$S_\ell$}}
	    \multiput(40,19)(20,0){3}{\cb{\color{green}$\bullet$}}\put(27,19){\cb{\color{green}$S_\ell$}}
	    \put( 20,4){\cb{$x_{j-2}$}}
	    \put( 40,4){\cb{$x_{j-1}$}}
	    \put( 80,4){\cb{$x_{j+1}$}}
	    \put(100,4){\cb{$x_{j+2}$}}
	    \put( 50,-3){\cb{\color{cyan}$x_{\jmh}$}}
	    \put( 70,-3){\cb{\color{green}$x_{\jph}$}}
	    \thicklines
	    \put(50,0){\color{cyan}\line(0,1){25}}
	    \put(70,0){\color{green}\line(0,1){25}}
	\end{picture}
    \end{center}
    \caption{<++>}<++>
    \label{fig:pQUICKfhStenL}
\end{figure}

If the basic variable is constrained to nodal value, %%%%% TBD %%%%%


\subsubsection{Construction of QUICK Scheme}


\subsubsection{The Properties of Interpolation Coefficient with Nodal Value as Basic Variable}\label{sec:prop-coeff-nodal-value}





\section{Absolute Error of High Order Schemes}
\label{sec:abserr-high-order}




\section{Conclusion}
\label{sec:conclusion}




% BiBTeX settings
\nocite{*}
\bibliography{report}

\appendix
\section{Proof of Properties of Interpolation Coefficients}\label{sec:proof}

First we define 

\end{document}

\if\false
%% How to add figures side by side?
\begin{figure}
    \centering
    \begin{subfigure}{.5\textwidth}
	\centering
	\includegraphics[width=.4\linewidth]{image1}
	\caption{A subfigure}
	\label{fig:sub1}
    \end{subfigure}%
    \begin{subfigure}{.5\textwidth}
	\centering
	\includegraphics[width=.4\linewidth]{image1}
	\caption{A subfigure}
	\label{fig:sub2}
    \end{subfigure}
    \caption{A figure with two subfigures}
    \label{fig:test}
\end{figure}

\begin{figure}
    \centering
    \begin{minipage}{.5\textwidth}
	\centering
	\includegraphics[width=.4\linewidth]{image1}
	\captionof{figure}{A figure}
	\label{fig:test1}
    \end{minipage}%
    \begin{minipage}{.5\textwidth}
	\centering
	\includegraphics[width=.4\linewidth]{image1}
	\captionof{figure}{Another figure}
	\label{fig:test2}
    \end{minipage}
\end{figure}

%% The simple manual for minted package
\begin{minted}[
	frame=lines,
	framesep=2mm,
	baselinestretch=1.2,
	bgcolor=LightGray,
	fontsize=\footnotesize,
    linenos]
    {python}
  % some Python code
\end{minted}

\inputminted[firstline=2, lastline=12]{octave}{BitXorMatrix.m}

\mint{html}|<h2>Something <b>here</b></h2>|

%% End %% Actually there are many fabulous features...
\fi

